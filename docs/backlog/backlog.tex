\documentclass{article}

% Like they did in the days of the old
\usepackage{fontspec}
\usepackage{OldStandard}

\usepackage{glossaries}

\makeglossaries

\newglossaryentry{board}
{
    name=Board,
    description={Refers to the main screen of our application, contains lists which in turn contain entries.}
}

\newglossaryentry{list}
{
    name=List,
    description={Are on the horizontal axis encapsulate entries.}
}

\newglossaryentry{entry}
{
    name=Entry,
    description={Refers to a task added to the board, organised on the vertical axis under a specific list.}
}

\usepackage{nopageno}

\usepackage{array}

\usepackage{svg}

\usepackage{amsmath}
\usepackage{amssymb}
\usepackage{caption}
\usepackage{fancybox}
\usepackage{xltxtra}

\usepackage[
  a4paper,
  right = 30mm,
  left = 30mm,
  bottom = 25mm,
  top = 30mm
]{geometry}

\usepackage{fancyhdr}
\usepackage{multicol}

\pagestyle{fancy}
\fancyhf{}
\lhead{\emph{Uruly Guitar}}
\rhead{Backlog}
\chead{CSE1105 Object-oriented Programming Project}
\rfoot{\thepage}



\begin{document}
	\vfill

	\begin{center}
    	\Large{Backlog - requirement engineering}
	\end{center}

	\section{Stakeholders}

	\begin{itemize}
    	\item \emph{User} - All users of our application, they are able to create a board using an id/key and modify it.
    	\item \emph{Admin} - A user that can unlock password protected boards, remove or add boards, and restart the server.
	\end{itemize}

	\section{Terminology}
	\vspace{-0.5cm}
	\printglossary[title={}]

	\section{The objective}

	We are building, as an overview, a personal task list organizer which is a board that contains lists and these lists in turn contain cards that can be moved between the lists and added/removed.

	\vfill
	\begin{center} \LARGE \textcolor{black!50!red!50!white}{\scshape INTENTIONALLY LEFT BLANK} \end{center}
	\vfill


	\clearpage

	\section{Requirements}

	\subsection{Mandatory for passing}

	\begin{enumerate}
		\item Create a Client-Server Application (Spring Boot + JavaFX)
		\item The boards must persist in a database, so the server should be connected to a database.
		\item The ability to create a new board.
		\item The ability to join a board created by another user.
		\item The ability to add/remove/edit cards and lists.
		\item The ability to pick a username before entering the project (no registration needed).
		\item A intuitive interface, options, and buttons (summarize titles + extra simple description where needed).
		\item The ability to have the tasks arbitrary ordered.
		\item The ability that the progress between all players is synchronized, so everybody sees the same stuff at the same time.
		\item The ability to drag and drop tasks.
		\item The ability to have the tasks easily recognizable based on the title.
	\end{enumerate}

	\subsection{Should be implemented}

	\begin{enumerate}
		\item The ability to join multiple boards using a key and a password and have them show up on the main page.
		\item The ability to specific the server I want to connect to.
		\item Password protected boards (will need to implement some hashing).
		\item The ability to change a board from a public one to a password protected board.
		\item Generation of SHA-1 identifiers for boards.
		\item To have details on cards such as a description.
		\item A nested task list.
		\item The ability to add different tags to an entry to classify the task.
		\item The ability customize boards and cards (Add a background, add colors for tags, choose font color and size).
		\item Easy keyboard shortcuts to facilitate use of the application.
		\item The ability to add sub tasks for tasks.
		\item The ability to add media to cards (Pictures, icons, and attachments).
		\item Easy edition of multiple tasks by virtue of the "tab" key.
		\item Editing tasks and sub tasks directly from the board an to have them highlighted when editing
		\item Different modes that a board can be set to (read-only, read-write).
	\end{enumerate}

	\subsection{Could be implemented}

	\begin{enumerate}
		\item A list of all boards.
		\item Board analytics such as: cards created, activity per week, users connected.
		\item Optional deadlines and progress bars on tasks.
		\item Filtering the tasks based on names and/or tags (optionally using fuzzy searching).
		\item Board history to see how a board has changed over time, with the authors of these changes.
	\end{enumerate}

	\subsection{Do not implement}
	 
	 \begin{enumerate}
	 	\item User management.
	 	\item Race condition remedies.
	 \end{enumerate}

	 \section{Mocks}

	 \vspace{0.5cm}

	 \begin{center}
	 	 \shadowbox{
	 	 	 \includegraphics[width=10cm]{mock1}}
	 	 \captionof{figure}{A mock task}
	 \end{center}
	 
	 \begin{center}
	 	 \shadowbox{
	 	 	 \includegraphics[width=10cm]{mock}}
	 	 \captionof{figure}{A mock of the board}
	 \end{center}


	 \lfoot{\small \XeLaTeX \hspace{0.1cm} 3.14}
	 
\end{document}
