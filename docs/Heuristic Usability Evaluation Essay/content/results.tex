% !TEX root =  ../results.tex
\section{Results}

Following the meeting with the other group for the Heuristic Usability Evaluation, we have asked each of their members to fill out a form based of questions on different heuristics regarding our application prototype. All the filled-out forms have been considered while making the following list. The different problems have been categorized and ordered/prioritized in terms of severity – from urgently needed to "can wait".

\subsection{System}

\begin{itemize}
    \item Multiple instances of error pop-ups, that the user might not know the origin and/or meaning of, would affect the "Match between system and the real world" heuristic.
    \begin{itemize}
        \item 400 HTTP error during list creation with an Empty title field
        \item DeploymentException error when entering an invalid hostname
        \item Sometimes the app shows system status messages
    \end{itemize}
    \item Connection to the server
    \begin{itemize}
        \item The 8080 port shouldn’t be entered by the user after the hostname
        \item Connects with ‘localhost’ but not with 127.0.0.1
    \end{itemize}
\end{itemize}

\subsection{Client}

\begin{itemize}
    \item A couple of functionalities not working as intended, affecting "Error prevention":
    \begin{itemize}
        \item Editing the Board name
        \item Board settings page
        \item Editing and deleting Lists
        \item Editing and deleting Cards
        \item Creating Cards
        \begin{itemize}
            \item Can’t add Sub-Tasks
            \item Non-time values can be entered
            \item Non-existent locations can be entered
        \end{itemize}
    \end{itemize}
    
    \item The following issues affect the "User control and Freedom" heuristic:
    \begin{itemize}
        \item No cancel button when adding a card
        \item It is not clear whether I’m logged in as an admin or not
        \item Can log in with the same user on multiple instances on the same computer
        \item No back button from Board Settings page
        \item Not many keyboard shortcuts, only TAB and SHIFT + TAB
        \item Login screen fields do not get cleared when leaving the page and coming back
        \item A user should not see the full Board ID in the ‘previous boards’ list, but their names
        \item Tags, location, and time aren't visible after adding a card
        \item No character limit to the names and descriptions of Board, CardList and Card
        \item Some documentation should be provided as it is hard for users to get around some features like, for ex., adding a card.
    \end{itemize}
\end{itemize}

\subsection{Visual}
Visual issues affect the "Aesthetic and minimalist design" heuristic.
\begin{itemize}
    \item Overall not satisfied with the visual aspect of the application:
    \begin{itemize}
        \item Very specific theme and bright colors that give out a blurry feeling
        \item Some buttons, like the back button on the Board View screen, are hard to spot
        \item List of previous boards has a different theme than the rest of the app
        \item Add List feature has a different theme than the rest of the app
    \end{itemize}
        \item Resizing:
    \begin{itemize}
        \item App switches between Fullscreen and windowed mode automatically often and unnecessarily
        \item Sometimes when automatically full screened, the ‘Press EXIT to get out of Fullscreen mode’ persists and doesn’t remove itself
    \end{itemize}
        \item Recommended color coding:
    \begin{itemize}
        \item Color-code adding and removing cards for minimal confusion
        \item Color-code the color choices in Board Settings page
    \end{itemize}
\end{itemize}

With these issues, we can create the following prioritization matrix based on their impact on the application and the frequency they appear in.

 \begin{figure}[ht]
    \centering
    \includegraphics[scale=0.65]{images/Matrix.pdf}
    \caption{Prioritization matrix}
    \label{fig:my_label}
\end{figure}

After thorough analysis, testing in a timely manner and, use of the previously mentioned matrix, we have concluded to opt for the following changes within our application: