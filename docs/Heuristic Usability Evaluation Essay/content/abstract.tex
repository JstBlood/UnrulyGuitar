This essay was written as a report concerning the Heuristic Usability Evaluation of the product "Unruly Guitar", conducted in independent runs by heuristic usability experts.

The product, "Unruly Guitar", is, as of the time of writing, a Spring boot application prototype which uses Gradle and has a JavaFX GUI. It is being developed for academic purposes, by students at the Technical University of Delft. The product seeks to provide aid for the collaboration of scale-able teams, using join-able boards with multiple TODO lists, which include outlined tasks.

The central question of the evaluation was how usable the product was for users of any level of technical knowledge, as those users constitute the main target group of the product. The evaluators were low-level experts in the field, as they themselves had also been developing their own JavaFX application with the same purpose as "Unruly Guitar". Furthermore, they were also well-versed with the process of Heuristic Usability Evaluation, having studied it prior to the evaluation.

The evaluators were supervised by M. Gazeel and V. Drăgutoiu, as to ensure their reports were in accordance with several usability heuristics that they outlined. Individual findings were analysed and aggregated into a collective report by M. Berzins and B. Micu.
