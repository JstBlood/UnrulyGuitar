% !TEX root =  ../methods.tex
\section{Methods}

The Heuristic Usability Report itself was conducted in alignment with the method detailed in the article \emph{How to Conduct A Heuristic Evaluation}, Nielsen et. al. \cite{how-to-coduct-a-heuristic-evaluation}. In anticipation to the evaluation, Group 22, the creators of "Unruly Guitar", compiled a list of recognised usability principles \cite{list-of-heuristics}, which served as the heuristics in question. These are:

\begin{enumerate}
    \item User control and freedom.
    \item Consistency and standards.
    \item Error prevention.
    \item Recognition rather than recall.
    \item Flexibility and efficiency of use.
    \item Aesthetic and minimalist design.
    \item Visibility of system status.
    \item Match between system and the real world.
    \item Help users recognize, diagnose, and recover from errors.
    \item Help and documentation.
\end{enumerate}

Thereafter, three experts were asked to volunteer to evaluate the product, again in accordance with the article. The subject of the evaluation was the product itself, as opposed to a mock-up or a paper-only prototype, as the product at the time of the evaluation was deemed sufficiently functional to be evaluated. The output of the evaluation consisted of three lists, one per evaluator, which provided succinct explanations of how the product related to each separate heuristic, with special care given to violations of the heuristics. Then, they were aggregated and analysed by members of Group 22.

\subsection{Experts}

The experts evaluating our application were:

\begin{itemize}
    \item Dorian Herbiet
    \item Teun Bosch
    \item Alex Tabacaru
    \item Rares Iordan
\end{itemize}

The experts were members of a different group from our course, four of them evaluated our project, and provided us with feedback. These experts, due to also working on an application with a similar goal and utilizing the same frameworks, were well acquainted with most of the inner workings of the system and had a general idea of how the system works.


\subsection{Procedure}

Two side-by-side instances of the client facing part of our system were prepared on one machine and presented to each expert. The experts were notified that there are three servers running, two of which were servers running our application whilst one of them was a server running an incompatible application, they were given the corresponding three IP addresses.

The experts were not instructed in any way shape or form as to how the application works or what they should expect, the only information they were provided was the aforementioned IP addresses.

The experts all followed a similar pattern of actions in order:

\begin{itemize}
    \item Tried to connect to a server.
    \item Joined a board.
    \item Added a list.
    \item Tried to add tasks to that list.
    \item Tried to connect to the other server.
    \item Tried to connect with the other client to witness client synchronization.
\end{itemize}

The experts were using the heuristics mentioned at the beginning of section 2.

\subsection{Measures}

The experts where each provided a form filled with a rephrased version of our heuristics. For each of these they were asked where the following is implemented or violated, how they have come upon this discovery, and/or how they would see a solution to improve on this in any way. This completed form, as sent in by 4 evaluators was our raw data for this report.