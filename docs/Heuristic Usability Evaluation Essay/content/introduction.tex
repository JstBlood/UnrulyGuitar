% !TEX root =  ../report.tex
\section{Introduction}

\subsection{Evaluation objective}
The target group of the product consists of teams, from any field, that want to coordinate a collaborative effort. These people might not necessarily be tech-savvy, and should preferably be able to use the product and all of its core features without having to sift through a lengthy documentation or tutorial. As such, ease-of-use and an intuitive interface were highly prioritised by Group 22, the developers of "Unruly Guitar".

Before the evaluation, Group 22 had dedicated a certain amount of effort towards achieving these goals. This effort was realised as a collection of user stories and epics that steered the team during development. These user stories would describe the intended user experience, so as to develop the product from the perspective of a user, keeping in mind the low level of entry regarding technical knowledge. However, specialised expertise regarding user usability was absent, and the product was not well polished yet, so flaws were expected to an extent. In addition, Group 22 had not reviewed the usability principles referenced earlier in this essay during development, only being made aware of them in anticipation of the evaluation.

The main goal of the evaluation was to amend these flaws, by analysing the report, and translating them to concrete goals that would be added to the backlog of the developers. Group 22 sees usability as integral to the product, as the target group of the product would be largely lost if were it not up to par.
\subsection{Prototype}

The product in question consists mainly of a board overview screen, where users can organise tasks, represented as "cards", into lists. These lists could be used as "To Do", "Doing" , "Done" or any other category the user found fitting, in the style of Trello \cite{Trello}, but to allow users flexibility, this is not necessary, and users can add, remove, and name their lists in any way they see fit. Users can also add tags, sub-tasks, descriptions, and other potential addenda not yet implemented at the time of writing to the cards. The boards can be accessed in parallel by all members of the board. Editing cards and organising them into lists constitutes the core functionality of the application.

On a more slightly higher level, the boards have a unique ID that users can use to join them, so whole teams can have access to the same board - in this sense, a board simulates a real-life whiteboard in a meeting room. A board can be renamed, and several customisation options exist to alter the appearance of a board. Furthermore, it can be optionally password protected, which bars any users that do not have access to the password from joining. Users can also choose to leave boards if they so wish.

Boards are hosted on servers, which can have admins, authenticated with admin passwords. An admin has the power to see all the boards hosted on a server, as opposed to a user, who can only see the boards they previously joined. An admin can also delete boards.

The UI design of the board seeks to emulate the idea of a blackboard, with a black background and a high-contrast and bright palette. The UI can also be interacted with without use of the mouse, as all elements are traverse-able using the TAB key. Generous highlighting and "go back"-buttons provide the user with instant visual feedback on their actions.
